% The document class supplies options to control rendering of some standard
% features in the result.  The goal is for uniform style, so some attention 
% to detail is *vital* with all fields.  Each field (i.e., text inside the
% curly braces below, so the MEng text inside {MEng} for instance) should 
% take into account the following:
%
% - author name       should be formatted as "FirstName LastName"
%   (not "Initial LastName" for example),
% - supervisor name   should be formatted as "Title FirstName LastName"
%   (where Title is "Dr." or "Prof." for example),
% - degree programme  should be "BSc", "MEng", "MSci", "MSc" or "PhD",
% - dissertation title should be correctly capitalised (plus you can have
%   an optional sub-title if appropriate, or leave this field blank),
% - dissertation type should be formatted as one of the following:
%   * for the MEng degree programme either "enterprise" or "research" to
%     reflect the stream,
%   * for the MSc  degree programme "$X/Y/Z$" for a project deemed to be
%     X%, Y% and Z% of type I, II and III.
% - year              should be formatted as a 4-digit year of submission
%   (so 2014 rather than the academic year, say 2013/14 say).

\documentclass[oneside,% the name of the author
                    author={Malak Hajji},
                % the degree programme: BSc, MEng, MSci or MSc.
                    degree={BSc},
                % the dissertation    title (which cannot be blank)
                    title={Designing An Accessible Computational Toolkit For Students},
                % the dissertation subtitle (which can    be blank)
                  subtitle={With Mixed Visual Abilities}]{dissertation}

\begin{document}

% =============================================================================

% This section simply introduces the structural guidelines.  It can clearly
% be deleted (or commented out) if you use the file as a template for your
% own dissertation: everything following it is in the correct order to use 
% as is.

\section*{Prelude}
\thispagestyle{empty}

A typical dissertation will be structured according to (somewhat) standard 
sections, described in what follows.  However, it is hard and perhaps even 
counter-productive to generalise: the goal is {\em not} to be prescriptive, 
but simply to act as a guideline.  In particular, each page count given is
important but {\em not} absolute: their aim is simply to highlight that a 
clear, concise description is better than a rambling alternative that makes
it hard to separate important content and facts from trivia.

You can use this document as a \LaTeX-based~\cite{latexbook1,latexbook2} 
template for your own dissertation by simply deleting extraneous sections
and content; keep in mind that the associated {\tt Makefile} could be of
use. %, in particular because it automatically executes \mbox{\BibTeX} to 
deal with the associated bibliography. 
Alternatively, upload this template, dissertation.bib, dissertation.cls, 
dtklogos.sty and the "logo" folder to Overleaf (an online \LaTeX editor and compiler) and work on your thesis there.

\textbf{Do not include this section in your final dissertation --- just delete it from the source.}

% =============================================================================

% This macro creates the standard UoB title page by using information drawn
% from the document class (meaning it is vital you select the correct degree 
% title and so on).

\maketitle

% After the title page (which is a special case in that it is not numbered)
% comes the front matter or preliminaries; this macro signals the start of
% such content, meaning the pages are numbered with Roman numerals.

\frontmatter


%\lstlistoflistings

% The following sections are part of the front matter, but are not generated
% automatically by LaTeX; the use of \chapter* means they are not numbered.

% -----------------------------------------------------------------------------

\chapter*{Abstract}

{\bf A compulsory section, of at most 300 words} 
\vspace{1cm} 

\noindent
This section should pr\'{e}cis the project context, aims and objectives,
and main contributions (e.g., deliverables) and achievements; the same 
section may be called an abstract elsewhere.  The goal is to ensure the 
reader is clear about what the topic is, what you have done within this 
topic, {\em and} what your view of the outcome is.

The former aspects should be guided by your specification: essentially 
this section is a (very) short version of what is typically the first 
chapter. If your project is experimental in nature, this should include 
a clear research hypothesis.  This will obviously differ significantly
for each project, but an example might be as follows:

\begin{quote}
My research hypothesis is that a suitable genetic algorithm will yield
more accurate results (when applied to the standard ACME data set) than 
the algorithm proposed by Jones and Smith, while also executing in less
time.
\end{quote}

\noindent
The latter aspects should (ideally) be presented as a concise, factual 
bullet point list.  Again the points will differ for each project, but 
an might be as follows:

\begin{quote}
\noindent
\begin{itemize}
\item I spent $120$ hours collecting material on and learning about the 
      Java garbage-collection sub-system. 
\item I wrote a total of $5000$ lines of source code, comprising a Linux 
      device driver for a robot (in C) and a GUI (in Java) that is 
      used to control it.
\item I designed a new algorithm for computing the non-linear mapping 
      from A-space to B-space using a genetic algorithm, see page $17$.
\item I implemented a version of the algorithm proposed by Jones and 
      Smith in [6], see page $12$, corrected a mistake in it, and 
      compared the results with several alternatives.
\end{itemize}
\end{quote}

% -----------------------------------------------------------------------------


\chapter*{Dedication and Acknowledgements}

{\bf A compulsory section}
\vspace{1cm} 

\noindent
It is common practice (although totally optional) to acknowledge any
third-party advice, contribution or influence you have found useful
during your work.  Examples include support from friends or family, 
the input of your Supervisor and/or Advisor, external organisations 
or persons who  have supplied resources of some kind (e.g., funding, 
advice or time), and so on.

% =============================================================================

% After the front matter comes a number of chapters; under each chapter,
% sections, subsections and even subsubsections are permissible.  The
% pages in this part are numbered with Arabic numerals.  Note that:
%
% - A reference point can be marked using \label{XXX}, and then later
%   referred to via \ref{XXX}; for example Chapter\ref{chap:context}.
% - The chapters are presented here in one file; this can become hard
%   to manage.  An alternative is to save the content in seprate files
%   the use \input{XXX} to import it, which acts like the #include
%   directive in C.

\mainmatter


% -----------------------------------------------------------------------------


\chapter*{COVID-19 Statement}

{\bf An optional section, of at most 800 words} 
\vspace{1cm} 

\noindent
A summary of any planned research activities disrupted by Covid-19 restrictions and the extent to which it was possible to adapt the work in those changed circumstances. If the project was able to go forward as planned, you can safely remove this section without losing any marks. The following may be included:

\begin{itemize}
\item Details of any planned research activities curtailed by the pandemic because of, for example, lack of access to facilities, libraries, archives, research participants, fieldwork, etc. Information on any curtailed training should be included only insofar as it relates to the impact on research activities and on the dissertation.

\item An acknowledgement of the anticipated contribution and value to the dissertation if those research activities had not been curtailed and what was possible to include in the dissertation in the circumstances, including where alternative choices were made to adapt the work and whether there are any weaknesses that could not be overcome.

\item Any other relevant factors on the impact of Covid-19 on research activities and on the contents of the dissertation.

\item Details of any research activities required by the examiners as part of a resubmission that were curtailed by the pandemic may be included in a new or revised Covid-19 statement in the resubmitted dissertation.
\end{itemize}

% -----------------------------------------------------------------------------

% This macro creates the standard UoB declaration; on the printed hard-copy,
% this must be physically signed by the author in the space indicated.

\makedecl



% -----------------------------------------------------------------------------

% LaTeX automatically generates a table of contents, plus associated lists 
% of figures and tables.  These are all compulsory parts of the dissertation.

\tableofcontents
\listoffigures
\listoftables

% -----------------------------------------------------------------------------



\chapter*{Ethics Statement}

{\bf A compulsory section} 
\vspace{1cm} 

In almost every project, this will be one of the following statements:
    \begin{itemize}
        \item ``This project did not require ethical review, as determined by my supervisor, [fill in name]''; or
        \item ``This project fits within the scope of ethics application 0026, as reviewed by my supervisor, [fill in name]''; or
        \item ``An ethics application for this project was reviewed and approved by the faculty research ethics committee as application [fill in number]''.
    \end{itemize}
    
See Section 3.2 of the unit Handbook for more information. If something went wrong and none of those three statements apply, then you should instead explain what happened.


% -----------------------------------------------------------------------------

\chapter*{Summary of Changes}

{\bf A conditional section} 
\vspace{1cm} 

If and only if the dissertation represents a resubmission (e.g., as the result of
a resit), this section is compulsory: the content should summarise all
non-trivial changes made to the initial submission.  Otherwise you can
omit it, since a summary of this type is clearly nonsensical.

When included, the section will ideally be used to highlight additional
work completed, and address criticism raised in any associated feedback.
Clearly it is difficult to give generic advice about how to do so, but
an example might be as follows:

\begin{quote}
\noindent
\begin{itemize}
\item Feedback from the initial submission criticised the design and 
      implementation of my genetic algorithm, stating ``there seems 
      to have been no attention to computational complexity during the
      design, and obvious methods of optimisation are missing within
      the resulting implementation''.  Chapter $3$ now includes a
      comprehensive analysis of the algorithm, in terms of both time
      and space.  While I have not altered the algorithm itself, I
      have included a cache mechanism (also detailed in Chapter $3$)
      that provides a significant improvement in average run-time.
\item I added a feature in my implementation to allow automatic rather
      than manual selection of various parameters; the experimental
      results in Chapter $4$ have been updated to reflect this.
\item Questions after the presentation highlighted a range of related
      work that I had not considered: I have make a number of updates 
      to Chapter $2$, resolving this issue.
\end{itemize}
\end{quote}

% -----------------------------------------------------------------------------

\chapter*{Supporting Technologies}

{\bf An optional section}
\vspace{1cm} 

\noindent
This section should present a detailed summary, in bullet point form, 
of any third-party resources (e.g., hardware and software components) 
used during the project.  Use of such resources is always perfectly 
acceptable: the goal of this section is simply to be clear about how
and where they are used, so that a clear assessment of your work can
result.  The content can focus on the project topic itself (rather,
for example, than including ``I used \mbox{\LaTeX} to prepare my 
dissertation''); an example is as follows:

\begin{quote}
\noindent
\begin{itemize}
\item I used the Java {\tt BigInteger} class to support my implementation 
      of RSA.
\item I used a parts of the OpenCV computer vision library to capture 
      images from a camera, and for various standard operations (e.g., 
      threshold, edge detection).
\item I used an FPGA device supplied by the Department, and altered it 
      to support an open-source UART core obtained from 
      \url{http://opencores.org/}.
\item The web-interface component of my system was implemented by 
      extending the open-source WordPress software available from
      \url{http://wordpress.org/}.
\end{itemize}
\end{quote}

% -----------------------------------------------------------------------------

\chapter*{Notation and Acronyms}

{\bf An optional section}
\vspace{1cm} 

\noindent
Any well written document will introduce notation and acronyms before
their use, {\em even if} they are standard in some way: this ensures 
any reader can understand the resulting self-contained content.  

Said introduction can exist within the dissertation itself, wherever 
that is appropriate.  For an acronym, this is typically achieved at 
the first point of use via ``Advanced Encryption Standard (AES)'' or 
similar, noting the capitalisation of relevant letters.  However, it 
can be useful to include an additional, dedicated list at the start 
of the dissertation; the advantage of doing so is that you cannot 
mistakenly use an acronym before defining it.  A limited example is 
as follows:

\begin{quote}
\noindent
\begin{tabular}{lcl}
AES                 &:     & Advanced Encryption Standard                                         \\
DES                 &:     & Data Encryption Standard                                             \\
                    &\vdots&                                                                      \\
${\mathcal H}( x )$ &:     & the Hamming weight of $x$                                            \\
${\mathbb  F}_q$    &:     & a finite field with $q$ elements                                     \\
$x_i$               &:     & the $i$-th bit of some binary sequence $x$, st. $x_i \in \{ 0, 1 \}$ \\
\end{tabular}
\end{quote}

% -----------------------------------------------------------------------------


\chapter{Introduction}
\label{chap:context}

{\bf Unlike the frontmatter up to and including the Summary of Changes, which you should not deviate from, Chapters 1--5 represent a suggested outline only. This outline will only be appropriate for a specific type of project. You should talk with your supervisor about the best way to structure your own dissertation, but ultimately the choice is yours. However, almost every project will want to include the content discussed in these chapters in some way. For more advice on structuring your dissertation, see the unit handbook.}
\vspace{1cm} 
\section{Problem}
\section{Approach}
\section{Aims}


\noindent
This chapter should introduce the project context and motivate each of the proposed aims and objectives.  Ideally, it is written at a fairly high-level, and easily understood by a reader who is technically competent but not an expert in the topic itself.

In short, the goal is to answer three questions for the reader.  First, what is the project topic, or problem being investigated?  Second, why is the topic important, or rather why should the reader care about it?  For example, why there is a need for this project (e.g., lack of similar software or deficiency in existing software), who will benefit from the project and in what way (e.g., end-users, or software developers) what work does the project build on and why is the selected approach either important and/or interesting (e.g., fills a gap in literature, applies results from another field to a new problem).  Finally, what are the central challenges involved and why are they significant? 
 
The chapter should conclude with a concise bullet point list that summarises the aims and objectives.  For example:

\begin{quote}
\noindent
The high-level objective of this project is to reduce the performance 
gap between hardware and software implementations of modular arithmetic.  
More specifically, the concrete aims are:

\begin{enumerate}
\item Research and survey literature on public-key cryptography and
      identify the state of the art in exponentiation algorithms.
\item Improve the state of the art algorithm so that it can be used
      in an effective and flexible way on constrained devices.
\item Implement a framework for describing exponentiation algorithms
      and populate it with suitable examples from the literature on 
      an ARM7 platform.
\item Use the framework to perform a study of algorithm performance
      in terms of time and space, and show the proposed improvements
      are worthwhile.
\end{enumerate}
\end{quote}

% -----------------------------------------------------------------------------

\chapter{Contextual Background}
\label{chap:technical}

\section{Learning Computational Thinking}  
\section{Visual Programming Languages}
\section{Non-Visual Programming Languages}
\section{Ozobot}

\noindent
This chapter is intended to describe the background on which execution of the project depends. This may be a technical or a contextual background, or both. The goal is to provide a detailed explanation of the specific problem at hand, and existing work that is relevant (e.g., an existing algorithm that you use, alternative solutions proposed, supporting technologies).  

Per the same advice in the handbook, note there is a subtly difference from this and a full-blown literature review (or survey).  The latter might try to capture and organise (e.g., categorise somehow) \emph{all} related work, potentially offering meta-analysis, whereas here the goal is simple to ensure the dissertation is self-contained.  Put another way, after reading this chapter a non-expert reader should have obtained enough background to understand what \emph{you} have done (by reading subsequent sections), then accurately assess your work against existing relevant related work.  You might view an additional goal as giving the reader confidence that you are able to absorb, understand and clearly communicate highly technical material and to situate your work within existing literature.


% -----------------------------------------------------------------------------

\chapter{Related Work}
\label{chap:execution}

\section{Block-based Tools for Learning}
\section{Robots for Learning}
\section{Discussion} 


\noindent
This chapter is intended to describe what you did: the goal is to explain
the main activity or activities, of any type, which constituted your work 
during the project.  The content is highly topic-specific, but for many 
projects it will make sense to split the chapter into two sections: one 
will discuss the design of something (e.g., some hardware or software, or 
an algorithm, or experiment), including any rationale or decisions made, 
and the other will discuss how this design was realised via some form of 
implementation.  

This is, of course, far from ideal for {\em many} project topics.  Some
situations which clearly require a different approach include:

\begin{itemize}
\item In a project where asymptotic analysis of some algorithm is the goal,
      there is no real ``design and implementation'' in a traditional sense
      even though the activity of analysis is clearly within the remit of
      this chapter.
\item In a project where analysis of some results is as major, or a more
      major goal than the implementation that produced them, it might be
      sensible to merge this chapter with the next one: the main activity 
      is such that discussion of the results cannot be viewed separately.
\end{itemize}

\noindent
Note that it is common to include evidence of ``best practice'' project 
management (e.g., use of version control, choice of programming language 
and so on).  Rather than simply a rote list, make sure any such content 
is useful and/or informative in some way: for example, if there was a 
decision to be made then explain the trade-offs and implications 
involved.


\begin{algorithm}[t]
\For{$i=0$ {\bf upto} $n$}{
  $t_i \leftarrow 0$\;
}
\caption{This is an example algorithm.}
\label{alg}
\end{algorithm}

\begin{lstlisting}[float={t},caption={This is an example listing.},label={lst},language=C]
for( i = 0; i < n; i++ ) {
  t[ i ] = 0;
}
\end{lstlisting}

This is an example sub-section;
the following content is auto-generated dummy text.
Notice the examples in Figure~\ref{fig}, Table~\ref{tab}, Algorithm~\ref{alg}
and Listing~\ref{lst}.
\lipsum

\subsubsection{Example Sub-sub-section}

This is an example sub-sub-section;
the following content is auto-generated dummy text.
\lipsum

\paragraph{Example paragraph.}

This is an example paragraph; note the trailing full-stop in the title,
which is intended to ensure it does not run into the text.

% -----------------------------------------------------------------------------

\chapter{Technical Background}
\label{chap:evaluation}



\noindent
In this chapter, I will be discussing the technologies I used while building my system. It involves a combination of:

\begin{enumerate}
\item Open-source fiducial marker generation and detection libraries.
\item OpenCV libraries.
\item Python GUI frameworks. 
\item Alternative Ozobot programming languages.
\end{enumerate}

\noindent
These elements all interact within the system to eventually program the Ozobot to complete a set of actions. This interaction occurs as follows.
 Firstly, the fiducial markers are placed on top of the physical programming blocks. It is important to note that each marker is unique to a block and the action depicted on that block. Secondly, the system uses OpenCV libraries to detect the markers on the block in real time video stream. The system has two different outputs depending on the action chosen. If the user chooses to translate the blocks in the image to action performed by the Ozobot, the detection algorithm returns a set of ID’s associated to each marker that my code translates to a list of actions. Finally, using an alternative Ozobot programming language called Ozopython, the system converts the list of actions to a sequence of flashing lights that the Ozobot can interpret and then perform the dictated actions. 
Alternatively, the user can choose the ‘Recognise Block’ action. This is a separate action that does not program the Ozobot, instead the system uses the OpenCV fiducial marker detection algorithm to detect the blocks and play an audio file describing the action that the block represents. 
Additionally, by using the Python GUI framework Tkinter, I am facilitating the user’s interaction with my system. All commands and information are accessible from the Tkinter window, and the buttons displayed on it. 
In this chapter, I aim to provide a general understanding of the open-source libraries and repositories I have used in my system and to justify the technical choices I have made, before providing an in-depth technical system design in chapter 5.

\section{Fiducial markers}
\noindent
The functioning of my system relies on the interpretation of the sequence of programming blocks. Hence, to ensure that the blocks are accurately detected, I have made use of fiducial markers. The camera is able to recognize the set of instructions through the AruCo marker encoding on each block. 
I have considered alternative approaches such as incorporating a barcode scanner in the block or using a shape detector, but all these approaches often provide inaccurate results. 
A fiducial marker system consists of a set of generated patterns that can be detected by a computer equipped with a camera and an appropriate marker detection algorithm.  Their design enables the markers to be physically small or be positioned at a large distance from the camera, while still being accurately detected in low-resolution images and at small sizes. Each marker features a unique ID number hence the marker IDs can be used to separate particular objects from others. In practice, these markers are placed in an environment to provide easily detectable visual cues and are used in applications like indoor tracking, robot navigation, augmented reality. These properties of fiducial markers make it a very suitable technology to use for block detection. 
Therefore, in this project, I am using fiducial markers to enable my system to detect the different programming blocks and create a flashing light sequence, which translates to a set of actions completed by the Ozobot. The markers also allow my system to execute the block recognition feature, where the system plays an audio file describing the action associated with the detected fiducial marker.  Before adopting fiducial markers in my system, I have considered different marker systems and eventually decided to employ arUco markers. In the following section, I will describe why the Aruco library is the most suitable for my system instead of other fidual marker systems.

\subsection{Types of Fiducial Marker Systems}
\subsubsection{}



\section{Image Processing} 
\subsection{imutils}

\noindent
Imutils are a series of functions to make image processing functions such as translation, rotation, resizing, skeletonization, and displaying images easier with OpenCV and both Python 2.7 and Python 3. Therefore, I am using imutils to manipulate live streamed images before detecting the markers using OpenCV. The following imutils functions are used:

\subsubsection{VideoStream}
With VideoStream, I have defined that source camera will be “0” which here corresponds to the webcam plugged to the computer port. After that, I have started the camera and defined that after start is initialized, the camera will turn on after 3 seconds so that camera sensor can warm up and image resolution is improved. From there, the .read() function is used to obtain the current frame captured by the webcam. The system continues looping over the frames of the live stream and computes live marker detection with 
OpenCV.

\subsubsection{Resize}
The resize function grabs the frame from the threaded video stream and resizes it to have a maximum width of 1000 pixels as the aruco detection algorithm requires.

\subsection{OpenCV}

To analyse the frames rendered by the imutils library, I have used the OpenCV library. OpenCV is a programming function library primarily used for Computer Vision applications. The three functions I have used to generate and detect markers obtained from my webcam are described here. 

\subsubsection{Drawmarker()}

Before their detection, ArUco markers need to be generated then printed in order to be placed in the environment. The OpenCV library has a built-in ArUco marker generator through its 
cv2.aruco.drawMarker function.The main parameters of this function is the marker dictionary  and the ID of the marker which has to be a valid ID In the ArUco dictionary. By specifying the marker dictionary and the marker ID, the drawMarker function then returns the output image with the ArUco marker drawn on it. 

There are 21 different ArUco dictionaries built into OpenCV. The majority of these dictionaries follow a specific naming convention, \texttt{cv2.aruco.DICT\_NxN\_M}, with an NxN size followed by an integer value, M. For example \texttt{DICT\_6X6\_250} means this dictionary is composed of 250 markers and a marker size of 6x6 bits. There are some settings to consider in order to choose the ideal dictionary, and these differ from system to system. It is generally more ideal to pick a dictionary with a larger NxN grid size, balanced with a low number of unique ArUco IDs such that the inter-marker distance can be used to correct misread markers. 
Therefore, for my system, I have chosen the \texttt{DICT\_7X7\_50}, this is because my system requires 15 unique IDs for 15 different actions, and this dictionary has the smallest number of unique IDs (50 IDs). The marker size of 7x7 bits also provides more accurate detection results than other marker sizes due to the larger grid size and inter-marker distance. 
I have generated a total of 15 markers for my system with ID’s in the range of [1,15], with one increment. 

\subsubsection{Detectmarker()}

The OpenCV library is used to detect the markers in the frames rendered by the imutils video library. I have made use of OpenCV’s cv2.aruco module specifically made to detect aruco markers. The marker detection process is comprised of two main steps:
\begin{enumerate}
    \item Detection of marker candidates. In this step, the frame read by the VideoStream.read() function described in 4.3.1. is analyzed in order to find square shapes that are candidates to be markers. It starts with an adaptive thresholding to segment the markers, then outlines are extracted from the thresholded image and those that do not approximate to a square shape are discarded. Some extra filtering is also applied to removing contours that are too small, too big, or too close to each other.
    \item After the candidate detection, it is necessary to determine if they are actually markers by analyzing their inner codification. This step starts by extracting the marker bits of each marker. The image is divided into different cells according to the marker size and the border size. Then the number of black or white pixels in each cell is counted to determine if it is a white or a black bit. Finally, the bits are analyzed to determine if the marker belongs to the specific dictionary. Error correction techniques are employed when necessary.
\end{enumerate}
I have designed my system to specifically detect markers from the dictionary \texttt{DICT\_7X7\_50} as specified in the previous paragraph. Since my system makes use of a live stream video from the webcam, I have included code to draw the outlines, the center and the ID of the detected tag on the screen to show real time detection. 
Finally, the IDs of the detected markers are stored in a list and used to translate to a flashing light sequence of instructions or parsed to the block recognition code, that plays an audio file depending on the block recognized. The working of this will be explained in more detail in the technical design chapter.

\section{playsound}

To implement the block recognition feature, I have used the playsound Python module. This enables my system to play mp3 audio files describing the action on the block detected by the OpenCV detection algorithm. My code performs this by detecting the marker ID, each marker ID has been assigned to an action. 
\begin{table}[t]
\centering
\begin{tabular}{|cc|c|}
\hline
Marker ID      & Action Associated      & Audio Output      \\
\hline
$1     $ & $\texttt{Start program} $ & $0     $ \\
$3     $ & $\texttt{Turn Left     }$ & $1     $ \\
$5     $ & $\texttt{U-turn Left}   $ & $  $\\
$7     $ & $\texttt{Move Forward}  $ & $9     $ \\
$9     $ & $\texttt{Variable Block: 5cm}  $ & $9     $ \\
\hline
\end{tabular}
\caption{Examples of markers and their corresponding actions.}
\label{tab}
\end{table}

\section{Tkinter}

To build my GUI, I have utilised the GUI Python Framework Tkinter and its functions to allow the user to interact with my system.

\subsection{Frame}
My GUI design involves a frame widget, it works on grouping and organizing other widgets by arranging their position. This widget is essential in my system since it allows me to choose the background color and the size of the displayed windows. For example, the parent window, which is the window displayed when the system starts, includes an organised layout of a label widget and different button widgets. The system displays a new frame depending on the buttons clicked. Hence, I have employed the frame widget as a foundation class to implement widgets and different windows.

\subsection{Button}
I have implemented the tkinter button widget in my system. There are two buttons that appear on the main window when the application starts. The buttons are linked to different functions in my code and perform the required actions depending on the button clicked.

\subsection{Label}
The label widget was used to display information such as descriptions or instructions at the top of the screen once the GUI has been loaded. I have also employed this widget to display any error messages that the system detects.

\section{Ozopy}
To interpret the detected markers and their associated instructions and convert them into flashing light sequence for the Ozobot, I have implemented an open-source Python-like language compiler for Ozobot [ref] called Ozopy, invented by Kaarl Ma..  This Python-like language is a subset of Python with some added builtin methods to control the Ozobot bit.

\subsection{Context}

The Ozobot can be programmed in two ways, either by using color codes and colored markers to indicate actions, or by using the visual programming interface OzoBlockly. With OzoBlockly, the user can create programs for the Ozobot by dragging puzzle shaped blocks onto the programming platform. The on-screen blocks clip together to form a program. The OzoBlockly editor has five skills levels from pre-reader to master.
For my system’s purposes, I will be focusing on level 1 and level 2, which are pre-reader and beginner levels. This is due to there being very few accessible programming interfaces for younger demographics. The level 1 blocks feature simple shapes depicting movements such as arrows, while level 2 blocks feature the same blocks but with written statements instead of icons with the addition of loops. 
To convert the created program into instructions understood by the Ozobot, OzoBlocky displays the encoded values as a light sequence composed of 8 color constants: black, white, red, green, yellow, blue, magenta and cyan. This is scanned at the base of the Ozobot where the RGB colour sensors reside. 

The engineering of the colour codes for OzoBlockly is not publicly available. However, Kaarel Maidre and Ashley Feniello have successfully reverse engineerd the colour codes and created Ozopy and FlashForth, programming IDE for the Ozobot. The following section explains the working of Ozopy.  

\subsection{Encoding Values}
\section{Summary}
% -----------------------------------------------------------------------------
\chapter{Exploring Accessible Programming}
\section{Study}

% -----------------------------------------------------------------------------
\chapter{System Design}
\section{Architecture}
\section{Physical Design}
\section{Technical Design}





% -----------------------------------------------------------------------------
\chapter{Post Protoype User Study}

% -----------------------------------------------------------------------------

\chapter{Conclusion and Further Work}
\label{chap:conclusion}

\noindent
The concluding chapter of a dissertation is often underutilised because it 
is too often left too close to the deadline: it is important to allocate
enough attention to it.  Ideally, the chapter will consist of three parts:

\begin{enumerate}
\item (Re)summarise the main contributions and achievements, in essence
      summing up the content.
\item Clearly state the current project status (e.g., ``X is working, Y 
      is not'') and evaluate what has been achieved with respect to the 
      initial aims and objectives (e.g., ``I completed aim X outlined 
      previously, the evidence for this is within Chapter Y'').  There 
      is no problem including aims which were not completed, but it is 
      important to evaluate and/or justify why this is the case.
\item Outline any open problems or future plans.  Rather than treat this
      only as an exercise in what you {\em could} have done given more 
      time, try to focus on any unexplored options or interesting outcomes
      (e.g., ``my experiment for X gave counter-intuitive results, this 
      could be because Y and would form an interesting area for further 
      study'' or ``users found feature Z of my software difficult to use,
      which is obvious in hindsight but not during at design stage; to 
      resolve this, I could clearly apply the technique of Smith [7]'').
\end{enumerate}

% =============================================================================

% Finally, after the main matter, the back matter is specified.  This is
% typically populated with just the bibliography.  LaTeX deals with these
% in one of two ways, namely
%
% - inline, which roughly means the author specifies entries using the 
%   \bibitem macro and typesets them manually, or
% - using BiBTeX, which means entries are contained in a separate file
%   (which is essentially a databased) then inported; this is the 
%   approach used below, with the databased being dissertation.bib.
%
% Either way, the each entry has a key (or identifier) which can be used
% in the main matter to cite it, e.g., \cite{X}, \cite[Chapter 2}{Y}.
%
% We would recommend using BiBTeX, since it guarantees a consistent referencing style 
% and since many sites (such as dblp) provide references in BiBTeX format. 
% However, note that by default, BiBTeX will ignore capital letters in article titles 
% to ensure consistency of style. This can lead to e.g. "NP-completeness" becoming
% "np-completeness". To avoid this, make sure any capital letters you want to preserve
% are enclosed in braces in the .bib, e.g. "{NP}-completeness".

\backmatter

\bibliography{dissertation}

% -----------------------------------------------------------------------------

% The dissertation concludes with a set of (optional) appendicies; these are 
% the same as chapters in a sense, but once signaled as being appendicies via
% the associated macro, LaTeX manages them appropriatly.

\appendix

\chapter{An Example Appendix}
\label{appx:example}

Content which is not central to, but may enhance the dissertation can be 
included in one or more appendices; examples include, but are not limited
to

\begin{itemize}
\item lengthy mathematical proofs, numerical or graphical results which 
      are summarised in the main body,
\item sample or example calculations, 
      and
\item results of user studies or questionnaires.
\end{itemize}

\noindent
Note that in line with most research conferences, the marking panel is not
obliged to read such appendices. The point of including them is to serve as
an additional reference if and only if the marker needs it in order to check
something in the main text. For example, the marker might check a program listing 
in an appendix if they think the description in the main dissertation is ambiguous.

% =============================================================================

\end{document}
